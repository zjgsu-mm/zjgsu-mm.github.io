\documentclass[mathserif, table]{beamer}

\usetheme{Boadilla}

\usepackage{subfig}
\usepackage{multirow}
\usepackage{xcolor,colortbl}
\usepackage{listings}
\usepackage{amsmath}

\lstset{language=C,tabsize=4, keepspaces=true,
    xleftmargin=2em,xrightmargin=2em, aboveskip=1em,
    backgroundcolor=\color{lightgray},    % 定义背景颜色
    frame=none,                      % 表示不要边框
    keywordstyle=\color{blue}\bfseries,
    breakindent=22pt,
    numbers=left,stepnumber=1,numberstyle=\tiny,
    basicstyle=\footnotesize,
    showspaces=false,
    flexiblecolumns=true,
    breaklines=true, breakautoindent=true,breakindent=4em,
    escapeinside={/*@}{@*/}
}

% Chinese
\usepackage{ctex}
\setCJKmainfont[BoldFont={SimHei},ItalicFont={KaiTi}]{KaiTi}

% Title
\title{Matlab实验二}
\author{韩建伟}
\institute{
  浙江工商大学信息学院\\
  \texttt{hanjianwei@mail.zjgsu.edu.cn}
}
\date{2011/10/31}

\begin{document}

% Title page
\begin{frame}[plain]
  \titlepage{}
\end{frame}

\begin{frame}{Matlab解线性方程}
  对于线性方程而言,可以写成$Ax=b$的形式,其解为:
  \[
  x = inv(A)b
  \]
  
  在进行求解之前,首先要保证$A$是方阵并且是可逆的,我们可以通过$rank(A)$求出$A$的秩,当$rank(A) == size(A,1) == size(A,2)$时方程才可以用上述方式求解。否则,我们要用下述方式求出其最小二乘解:

  \[
  x = inv(A'A)A'b
  \]

  习题:试用最小二乘解来求解课本107页的用二次多项式拟合带式录音机的录音时间问题

\end{frame}

\begin{frame}{Matlab多项式表示}
  在Matlab中,$n$次多项式可以用一个$n+1$维的向量来进行表示,比如$p1(x) = x^3-2x^2+5x+3, p2(x)=6x-1$在Matlab中分别表示为:

  \begin{block}{}
    p1 = [1, -2, 5, 3];

    p2 = [0, 0, 6, -1];

    c = p1 + p2; \% 多项式加法

    d = conv(p1, p2); \% 多项式乘法

    [q, r] = deconv(p1, p2); \% 多项式除法, q是商式,r是余式

    p = polyder(p1); \% 多项式p1的导函数

    p = polyder(p1, p2); \% 多项式p1和p2乘积的导函数

    [p, q] = polyder(p1, p2); \% p1,p2之商的导函数,p,q分别是导函数的分子、分母

    y = polyval(p1, 4); \% 多项式求值,p1在x=4处的函数值
  \end{block}
  
\end{frame}

\begin{frame}{多项式拟合}
  polyfit可以根据给定的点确定一个$n$阶多项式:

  \begin{block}{}
    x = [1.1, 2.3, 3.9, 5.1]; 

    y = [3.887, 4.276, 4.651, 2.117];

    a = polyfit(x, y, length(x)-1);  \% 拉格朗日插值

    b  = polyfit(x, y, 2);  \% 2阶多项式拟合
  \end{block}
  
  试用polyfit对105页表4-12的三组数据分别进行不同阶数的多项式拟合,并画出函数图。

\end{frame}

\begin{frame}{样条插值}
  在Matlab中,可以用interpl进行插值yi=interp1(x0,y0,xi,'method'),其中x0,y0是数据点,xi是需要求值的点,yi是插值得到的值。method如果是linear表示线性内插,如果是cubic是三次多项式内插,如果是spline是三次样条内插,如果是nearest是最近点内插,method的左上角如果加有星号*表示将插值区间等距的分割。

  \begin{block}{}
    x0=[0.00 0.20 0.40 0.60 0.80 1.00];

    y0=[0.00 0.32 0.21 0.10 0.04 0.02];

    x=0:0.03:1;

    y1=interp1(x0,y0,x);

    y2=interp1(x0,y0,x,'*cubic');

    y3=interp1(x0,y0,x,'*spline');

    plot(x0,y0,'ko',x,y1,'b:',x,y2,'r-.',x,y3,'m-')
  \end{block}

  试着用不同的插值方式对120页表4-26的数据进行插值
  
\end{frame}

\begin{frame}{编程}
  学习网上教程,试着用蒙特卡罗方法计算$y=x^2$在[2,3]的积分
\end{frame}


\end{document}

%%% Local Variables: 
%%% TeX-master: t
%%% TeX-engine: xetex
%%% End: 
